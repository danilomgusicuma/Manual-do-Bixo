% Este arquivo .tex será incluído no arquivo .tex principal. Não é preciso
% declarar nenhum cabeçalho

\section{Conpec}

\begin{figure}[H]
    \centering
    \includegraphics[width=.35\textwidth]{img/alem_da_graduacao/conpec_logo.png}
\end{figure}

A Conpec é a empresa júnior dos cursos de Ciência e Engenharia de Computação da
Unicamp. "Mas o que é uma empresa júnior?", você me pergunta, cara bixete ou bixo. 
É uma entidade,criada para diminuir o \textit{gap} entre a graduação e o mercado
de trabalho.
Ou seja, na Conpec você vai poder ir além daquele tempo que você fica na sala 
de aula e colocar computação na prática logo no seu primeiro ano de faculdade!

Aqui, você vai ter a oportunidade de adquirir e aplicar conhecimentos novos em 
situações reais, com clientes, prazos e projetos de impacto! Além disso, é uma 
chance enorme de aprender sobre aspectos do mercado de trabalho que você nunca 
veria na faculdade, como marketing, finanças, gestão de pessoas, metodologias de
desenvolvimento, trabalho em equipe e liderança, indispensáveis considerando-se que
o perfil empreendedor é cada vez mais exigido dos profissionais de computação.

Muitos membros e ex-membros da Conpec usam os conhecimentos adquiridos na
empresa não só como um adicional ao buscar uma vaga no mercado de trabalho, mas
também para montar suas próprias empresas e tocar seus projetos pessoais. Por isso,
os principais nomes do ecossistema de empresas de tecnologia de Campinas passaram por
aqui. Um exemplo disso é o Fabrício Bloisi, CEO da Movile, empresa líder da América Latina
em aplicativos e conteúdos para \textit{smartphones}. Dá uma olhada no que ele falou da
experiência na Conpec:
\begin{quote}
A Conpec foi um passo muito importante para a criação da Movile. Entrei na Conpec no 
meu primeiro ano do curso de Ciência da Computação e lá passei por muitas funções: 
Desenvolvedor, Gerente de Projetos, Diretor Comercial, Conselheiro. Foi esta experiência 
que me permitiu abrir a Movile logo no ano seguinte. Na Conpec tive experiências com 
tecnologia, gente, gestão, vendas, liderança. Com certeza acho a Empresa Junior uma excelente 
oportunidade para desenvolver o empreendedorismo - focado em abrir empresas ou em liderar 
negócios em empresas existentes. Um abraço e Boa sorte a todos da Conpec!
\end{quote}

Além disso, a Conpec é uma excelente oportunidade para conhecer seus colegas de curso,
sejam veteranos ou bixos, pessoas de outros cursos e até mesmo
de fora da Unicamp, uma vez que há diversas empresas juniores espalhadas pelas
universidades de São Paulo e do Brasil. É ainda uma grande
chance para perder a inibição de falar em público e aperfeiçoar sua capacidade de expor
opiniões, além de aprender como agir em um ambiente profissional.

\begin{figure}[H]
    \centering
    \includegraphics[width=.45\textwidth]{img/alem_da_graduacao/conpec_foto.jpg}
\end{figure}

Para fazer parte da Conpec, fique atent* à data da palestra de apresentação do
processo seletivo, que ocorre no início do ano.

Para saber mais sobre a empresa visite o site \url{conpec.com.br} ou tire suas
dúvidas mandando um e-mail para \email{conpec@conpec.com.br}.
